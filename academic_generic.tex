% Original Template Author: Rishi Shah

% Document Configuration
\documentclass{resume}
\usepackage[left=0.75in,top=0.6in,right=0.75in,bottom=0.6in]{geometry}
\newcommand{\tab}[1]{\hspace{.2667\textwidth}\rlap{#1}}
\newcommand{\itab}[1]{\hspace{0em}\rlap{#1}}
\name{Jack Gisby}
\address{2 Manilla Road, Birmingham, B29 7PU}
\address{(+44)7748 559964  \\ jackgisby@gmail.com \\ https://github.com/jackgisby}

% Document Content 
\begin{document}
\begin{rSection}{Research Interests}
Biochemist with an interest and aptitude for computational biology and programming. Developed my computational skills using online courses and self-directed projects; applied my knowledge to bioinformatics research projects, where I developed interests in data-driven research and the "omics" technologies. 

After comparing my experiences in industry and academia, it is clear that I thrive in an academic environment at the frontier of human knowledge; therefore, I would like to take on a PhD project where I may conduct my own research and contribute to my fields of interest. Following this, I will likely pursue a post-doctoral position. 
\end{rSection}
\begin{rSection}{Education}

{\bf University of Birmingham} \hfill {\em September 2016 - July 2020} 
\\ Biochemistry with Professional Placement (MSci) \hfill {Current Average: 74\%} \smallskip \\
Key courses: Gene Expression Analysis, Genetics, Research and Funding, Experimental Design and Analysis. Significantly improved my statistical and programming skills while gaining a background in molecular biology, physiology and genetics. \smallskip \\ Dissertation project involved the development and validation of a freely-available computational workflow, \textit{Metaboverse}.  \\

\end{rSection}
\begin{rSection}{Research Experience}

\begin{rSubsection}{A Prior Knowledge-Based Computational Workflow for \textit{de novo} Structural Elucidation of Small Molecules in Mass Spectrometry Metabolomics}{}{Dr Ralf Weber}{October 2019 - Present}

\item Developed my understanding of databases, data structures and algorithms for the optimisation of \textit{Metaboverse}; learned about NP-complete problems, graph theory and combinatorial optimisation for the representation and efficient generation of small molecules. 
\item Processed large, multi-dimensional mass spectrometry datasets for the validation of \textit{Metaboverse}. 
\item Produced a pipeline for the generation of large molecular databases using Python, SQLite and a Linux high performance computing cluster. 
\item Understood, and built on, an existing codebase. 
\item Developed machine learning models based on structural properties to assign a "metabolite-likeness" score to generated molecules. 
\end{rSubsection}

\begin{rSubsection}{An Automated Tool for \textit{de novo} Annotation of Pack-TYPE Non-Autonomous Transposable Elements}{}{Dr Marco Catoni}{August 2019 - Present}

\item Designed and implemented an R package to automatically detect transposable elements. 
\item Used command line and Bioconductor tools to process sequence data. 
\item Adhered to R standard practices for the organisation, testing and documentation of packages; learned to use version control for the management of programming projects. 
\item The package is currently being used by geneticists to understand how Pack-TYPE transposons may impact evolution; results generated from the tool will be presented at a conference in Shanghai.
\end{rSubsection}

\newpage
\begin{rSubsection}{Development of an Immunoturbidimetric Assay for Serum Amyloid A}{}{The Binding Site}{August 2018 - August 2019}

\item Routinely used Excel, and statistical packages such as Analyse-It/Minitab, to analyse and present data to co-workers and managers. 
\item Took responsibility for planning and carrying out experiments and statistical analyses in a research environment. 
\item Received training in scientific writing in order to routinely produce experimental write-ups for submission to regulatory bodies; this involved compliance to standard operating procedures.
\item Developed an assay for Serum Amyloid A for use in clinical laboratories; this long term project developed my ability to work autonomously.
\item Produced an extended literature review and project report before delivering a presentation at the University of Birmingham based on my experiences. 
\item Interacted with, and regularly delivered presentations to, researchers from diverse fields and executive members of the company to ensure the timely completion of my development project. 
\end{rSubsection}

\end{rSection}
\begin{rSection}{Specific Research Skills}

\begin{tabular}{ @{} >{\bfseries}l @{\hspace{6ex}} l }
Operating Systems \ & Windows, Ubuntu \\
Programming Languages \ & Python, R, SQLite \\
Related Tools \ & Bioconductor, ggplot2, SciPy, Matplotlib, sklearn, Samtools,\\ \ & Git, BLAST, Linux HPC \\
General Software \ & MS Office, Latex  \\
\end{tabular}

\end{rSection}
\begin{rSection}{Additional Courses} 

\item Lean Competency Level 1 Certification
\item The Biostar Handbook: Bioinformatics Data Analysis by Istvan Albert
\item Introduction to Computer Science and Programming Using Python by MITx
\item Introduction to Computational Thinking and Data Science by MITx
\item Data Analysis for Life Sciences and Genomics Data Analysis by HarvardX
\item Metabolomics: Understanding Metabolism in the 21st Century by University of Birmingham
\item Rosalind "Bioinformatics Stronghold" Problem Sets (Level 5 - Python)
\item Machine Learning, SQL and Competitions on Kaggle

\end{rSection}
\begin{rSection}{Additional Skills \& Interests} \itemsep -3pt

\item Regularly complete online courses and work on personal programming projects in my own time.  Have applied my learning to teach others during after school "Code Clubs", developing my own teaching ability and providing help with coding skills to younger children. 
\item Avid boulderer and beginner trad climber - climbing is excellent for fitness, very social and I enjoy supervising beginner climbers. 

\newpage
\end{rSection}
\begin{rSection}{References}

\begin{center}
\setlength{\tabcolsep}{13pt}
\begin{tabular}{ccc} 
Dr Klaus Futterer (Tutor) & Dr Marco Catoni & Dr Ralf Weber \\ 
Reader in Structural Biology & Lecturer in Plant Biology & Director of Bioinformatics \\ 
School of Biosciences & School of Biosciences & Phenome Centre Birmingham \\ 
University of Birmingham & University of Birmingham  & University of Birmingham \\
Edgbaston & Edgbaston & Edgbaston \\
B15 2TT & B15 2TT & B15 2TT \\
k.futterer@bham.ac.uk & m.catoni@bham.ac.uk & r.j.weber@bham.ac.uk \\
\end{tabular}
\end{center}
 
\end{rSection}
\end{document}
