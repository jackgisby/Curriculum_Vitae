%%%%%%%%%%%%%%%%%%%%%%%%%%%%%%%%%%%%%%%%%
% Medium Length Professional CV
% LaTeX Template
% Version 2.0 (8/5/13)
%
% This template has been downloaded from:
% http://www.LaTeXTemplates.com
%
% Original author:
% Rishi Shah 
%
% Important note:
% This template requires the resume.cls file to be in the same directory as the
% .tex file. The resume.cls file provides the resume style used for structuring the
% document.
%
%%%%%%%%%%%%%%%%%%%%%%%%%%%%%%%%%%%%%%%%%

%----------------------------------------------------------------------------------------
%	PACKAGES AND OTHER DOCUMENT CONFIGURATIONS
%----------------------------------------------------------------------------------------

\documentclass{resume}

\usepackage[left=0.75in,top=0.6in,right=0.75in,bottom=0.6in]{geometry} % Document margins
\newcommand{\tab}[1]{\hspace{.2667\textwidth}\rlap{#1}}
\newcommand{\itab}[1]{\hspace{0em}\rlap{#1}}
\name{Jack Gisby}
\address{2 Manilla Road, Birmingham, B29 7PU}
\address{(+44)7748 559964  \\ jackgisby@gmail.com \\ https://github.com/jackgisby} % Your phone number and email
% https://github.com/jackgisby

%----------------------------------------------------------------------------------------
%	CV Content
%----------------------------------------------------------------------------------------

\begin{document}
\begin{rSection}{Research Interests}

Highly motivated biochemistry student, with an interest in computational biology, seeking to develop my research and computational skills. The application of next generation sequencing (NGS) and other "omics" technologies to study complex biological systems is fascinating, and fraught with computational challenges. 

Having discovered a fascination for programming during my degree, I began learning about computer science and bioinformatics before applying my skills to the development of an R package for the identification of Pack-TYPE transposable elements; these elements acquire chromosomal DNA after incomplete excisions of closely spaced repeats and are poorly predicted in annotation studies. "packFinder" facilitates the detection of these transposons, which have important implications in the evolution of host genes. 

My dissertation currently involves development of the python package "Metaboverse" which attempts to expand the search space of small molecules for untargeted metabolomics experiments. Improved automated annotation of metabolites will improve the standard workflows for the growing field of metabolomics; improvements in high-throughput technologies will lead to the possibility of improved multi-omics integration.

After comparing these projects to my experience in industry, it is clear that I thrive in an academic environment at the frontier of human knowledge; therefore, I would like to pursue a PhD where I may conduct my own research and contribute to my fields of interest. As a biologist with a passion for computation and the "omics" technologies, I am particularly keen to develop my programming and technical skills. My skillset, previous research experience and tenacious nature will make me a commited and competent PhD candidate. 
 
\end{rSection}
\begin{rSection}{Education}

{\bf University of Birmingham} \hfill {\em September 2016 - Present} 
\\ Biochemistry with Professional Placement (MSci) \hfill {Current Average: 74\%} \smallskip \\
Key courses: Gene Expression Analysis, Genetics, Research and Funding, Experimental Design and Analysis. Significantly improved my statistical and programming skills while gaining a background in molecular biology, physiology and genetics. Dissertation project involved the development and validation of a freely-available computational workflow, "Metaboverse".  \\

\end{rSection}
\begin{rSection}{Research Experience}

\begin{rSubsection}{A Prior Knowledge-Based Computational Workflow for \textit{de novo} Structural Elucidation of Small Molecules in Mass Spectrometry Metabolomics}{}{Dr Ralf Weber}{October 2019 - Present}
\item Extended an existing codebase for the generation of \textit{de novo} metabolites and refactored for PEP8 compliance. Developed my understanding of databases, data structures and algorithms for the optimisation of Metaboverse; used isomorphic graphs for the generation of molecules from substructures stored in SQLite databases.
\item Processed mass spectrometry metabolomics datasets for the validation of Metaboverse. Produced a pipeline for the generation of large substructure databases using a linux high performance computing cluster. 
\end{rSubsection}

\begin{rSubsection}{An Automated Tool for \textit{de novo} Annotation of Pack-TYPE Non-Autonomous Transposable Elements}{}{Dr Marco Catoni}{August 2019 - Present}
\item Designed and implemented an R package to automatically detect transposable elements. Used command line and bioconductor tools to process sequence data. 
\item Adhered to R standard practices for the organisation, testing and documentation of packages; learned to use version control for the management of programming projects. 
\end{rSubsection}

\begin{rSubsection}{Development of an Immunoturbidimetric Assay for Serum Amyloid A}{}{The Binding Site}{August 2018 - August 2019}
\item Routinely used excel, and statistical packages such as Analyse-It/Minitab, to analyse and present data to co-workers and managers. 
\item Took responsibility for planning and carrying out experiments and statistical analyses in a research environment. Received training in scientific writing in order to routinely produce experimental write-ups for submission to regulatory bodies; this involved compliance to standard operating procedures.
\item Developed an assay for serum amyloid A for use in clinical laboratories; produced an extended literature review and project report before giving a presentation at the University of Birmingham based on my experiences. 
\end{rSubsection}

\end{rSection}
\begin{rSection}{Specific Research Skills}

\begin{tabular}{ @{} >{\bfseries}l @{\hspace{6ex}} l }
Operating Systems \ & Windows, Ubuntu \\
Programming Languages \ & Python, R, SQL \\
Related Tools \ & Bioconductor, ggplot2, SciPy, Matplotlib, sklearn, Samtools,\\ \ & Git, BLAST \\
General Software \ & MS Office, Latex  \\
\end{tabular}

\end{rSection}
\begin{rSection}{Additional Courses} 

\item The Biostar Handbook: Bioinformatics Data Analysis by Istvan Albert
\item Introduction to Computer Science and Programming Using Python by MITx
\item Introduction to Computational Thinking and Data Science by MITx
\item Data Analysis for Life Sciences and Genomics Data Analysis by HarvardX
\item Metabolomics: Understanding Metabolism in the 21st Century by University of Birmingham
\item Rosalind "Bioinformatics Stronghold" Problem Sets (Level 5 - Python)

\end{rSection}
\begin{rSection}{Additional Skills \& Interests} \itemsep -3pt

\item Regularly complete online courses and work on personal programming projects in my own time.  Have applied my learning to teach others during after school "Code Clubs", developing my own teaching ability and providing help with coding skills to younger children. 
\item Avid boulderer and beginner trad climber - climbing is excellent for fitness, very social and I enjoy supervising beginner climbers. 

\end{rSection}
\begin{rSection}{References}

Available upon request.
 
\end{rSection}
\end{document}
